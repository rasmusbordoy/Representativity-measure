% Options for packages loaded elsewhere
\PassOptionsToPackage{unicode}{hyperref}
\PassOptionsToPackage{hyphens}{url}
%
\documentclass[
]{article}
\usepackage{amsmath,amssymb}
\usepackage{iftex}
\ifPDFTeX
  \usepackage[T1]{fontenc}
  \usepackage[utf8]{inputenc}
  \usepackage{textcomp} % provide euro and other symbols
\else % if luatex or xetex
  \usepackage{unicode-math} % this also loads fontspec
  \defaultfontfeatures{Scale=MatchLowercase}
  \defaultfontfeatures[\rmfamily]{Ligatures=TeX,Scale=1}
\fi
\usepackage{lmodern}
\ifPDFTeX\else
  % xetex/luatex font selection
\fi
% Use upquote if available, for straight quotes in verbatim environments
\IfFileExists{upquote.sty}{\usepackage{upquote}}{}
\IfFileExists{microtype.sty}{% use microtype if available
  \usepackage[]{microtype}
  \UseMicrotypeSet[protrusion]{basicmath} % disable protrusion for tt fonts
}{}
\makeatletter
\@ifundefined{KOMAClassName}{% if non-KOMA class
  \IfFileExists{parskip.sty}{%
    \usepackage{parskip}
  }{% else
    \setlength{\parindent}{0pt}
    \setlength{\parskip}{6pt plus 2pt minus 1pt}}
}{% if KOMA class
  \KOMAoptions{parskip=half}}
\makeatother
\usepackage{xcolor}
\usepackage[margin=1in]{geometry}
\usepackage{graphicx}
\makeatletter
\def\maxwidth{\ifdim\Gin@nat@width>\linewidth\linewidth\else\Gin@nat@width\fi}
\def\maxheight{\ifdim\Gin@nat@height>\textheight\textheight\else\Gin@nat@height\fi}
\makeatother
% Scale images if necessary, so that they will not overflow the page
% margins by default, and it is still possible to overwrite the defaults
% using explicit options in \includegraphics[width, height, ...]{}
\setkeys{Gin}{width=\maxwidth,height=\maxheight,keepaspectratio}
% Set default figure placement to htbp
\makeatletter
\def\fps@figure{htbp}
\makeatother
\setlength{\emergencystretch}{3em} % prevent overfull lines
\providecommand{\tightlist}{%
  \setlength{\itemsep}{0pt}\setlength{\parskip}{0pt}}
\setcounter{secnumdepth}{-\maxdimen} % remove section numbering
\ifLuaTeX
  \usepackage{selnolig}  % disable illegal ligatures
\fi
\IfFileExists{bookmark.sty}{\usepackage{bookmark}}{\usepackage{hyperref}}
\IfFileExists{xurl.sty}{\usepackage{xurl}}{} % add URL line breaks if available
\urlstyle{same}
\hypersetup{
  pdftitle={Representativity measure - working paper},
  pdfauthor={Rasmus Bordoy, Thomas Gerds, Nina Føns Johnsen, Sidsel Marie Bernt Jørgensen},
  hidelinks,
  pdfcreator={LaTeX via pandoc}}

\title{Representativity measure - working paper}
\author{Rasmus Bordoy, Thomas Gerds, Nina Føns Johnsen, Sidsel Marie
Bernt Jørgensen}
\date{September 2023}

\begin{document}
\maketitle

In regards to studying participation in a survey by The Danish Heart
Foundation called Life With A Heart disease, then the following subject
has become of interest.

\newline

\textbf{Description of subject and theme}

The subject/theme is to find some measure that speaks of the
representativity amongst the participants in the survey. The notion of
representativity is somewhat arbitrary, since it does not have a clear
statistical meaning. In short, the interest is to investigate to what
extent the proportions of groups within a dataset (respondants only) is
equal or perhaps homogeneous.

\newline

Some representativity measures will be presented through examples, first
for categorical groups, and then later an attempt to incorporate
continuous variables into the measure will be made.

\newline

\textbf{Examples}

\textbf{Example 1} - \emph{Contribution of disjoint groups falling below
the overall response rate}

In this example the intention is to measure how much the response rate
within \textbf{two} groups differs from the overall response rate.

Assume that there are two stratifications in the data meaning that there
is only one covariate with two groups in the following setup.

The difference in response rate is measured whenever the response rate
falls below the \(\underline{\text{overall response rate}}\), since some
may argue that an over-representation within a group doesn't cause
problems nor more potential.

\newline

Let \({p}_{\text{overall}} = \frac{\sum_{i=1}^{N} \pi_i}{N}\) be the
overall response rate for a response indicator \(\pi_i \in \{0,1\}\).
Then let \(g_1\) and \(g_2\) be two disjoint groups within the data, say
male and female, and let \({p}_{g_1}\) and \({p}_{g_2}\) be their
response rates. The categorical response rate is defined as
\[{p}_{g_1} =  \frac{\sum_{i=1}^{N_{g_1}} \pi_i }{N_{g_1}},\] where
\(N_{g_1}\) is the number of records within group \(1\).

Allowing the contribution from response rates within groups only being
present whenever the response rates fall below \({p}_{\text{overall}}\),
then the following measure measures to which extent the response rates
fall below the overall response rate.

\begin{align}

\text{R} & = \frac{{p}_{g_1} 1_{  \{ {p}_{g_1} <{p}_{\text{overall}} \} }+
{p}_{\text{overall}} 1_{ \{{p}_{g_1} \geq {p}_{\text{overall}} \} 
}
\,\,\,\,+
\,\,\,\,
{p}_{g_2} 1_{ \{ 
{p}_{g_2} <{p}_{\text{overall}} \}
}+{p}_{\text{overall}} 
1_{ \{ {p}_{g_2} \geq {p}_{\text{overall}} \}} }
{2 \cdot {p}_{\text{overall}}} \\
& = 1+\frac{ \frac{1}{2} \left\{  1_{ \{ p_{g_1} < p_{\text{overall}} \} } \left( p_{g_1} - p_{\text{overall}} \right) \,\, + \,\,  1_{ \{ p_{g_2} < p_{\text{overall}} \} } \left( p_{g_2} - p_{\text{overall}} \right)  \right\}  }{p_{\text{overall}}},
\end{align} where
\(-1 \leq 1_{ \{ p_{g_1} < p_{\text{overall}} \} } \left( p_{g_1} - p_{\text{overall}} \right) \leq 0\)
leading to \(R \in [0,1]\). This leads to an average representativity of
the groups in terms of response, and it is measured from a perspective
that reaching \(1\), or say \(100 \%\), is only a possibility if the
response rates (not the number of responses) are perfectly split across
the groups.

\newline

A covariate with \(K\) groups has the following categorical
representativity measure

\begin{align}

\text{R} & = 
1+\frac{ \frac{1}{K} \left\{ \sum_{i=1}^K 1_{ \{ p_{g_i} < p_{\text{overall}} \} } \left( p_{g_i} - p_{\text{overall}} \right)  \right\}  }{p_{\text{overall}}}.
\end{align}

\textbf{Example 2} - \emph{Contribution of continuous covariate falling
below the overall response rate}

\newline Considering contributions of covariates that are continuous,
then a different strategy has to be considered in order to find the
representativity and balance though the values the covariate takes in
regards of response rates. Assuming the continuous covariate is age,
then it is of interest to consider how the response rates are
distributed along the values the age covariate takes and then measure
how this behaviour is compared to the overall response rate. Response
rates above the overall response rate are, again, assumed to not
contribute to the representativity any more than the overall response
rate, but contributions below the overall response rate will be
``punished'' in regards to lacking representativity.

The presented representativity measure will represent the balance in the
response for a continuous covariate, but it does not represent which
values in the covariate are either balanced or unbalanced. In other
words, the representativity of the continuous covariate is combined into
a single number representing the overall average representativity within
the continuous covariate.

Consider the following figure

\includegraphics{Representativity-measure---working-paper-PDF_files/figure-latex/unnamed-chunk-1-1.pdf}

In the figure above it is seen that the response rate for a given age on
the x-axis changes throughout. The overall response rate is \(0.55\),
whilst the age specific response rate follows some arbitrary given sine
function chosen for illustration.

The area below the line with age specific rates whilst being below the
overall rate will be the contribution of response rate from the age
covariate in general, and the area between the overall rate and age
specific rate whenever the latter is below the overall rate will be the
contribution to ``loss of response''. The following figure illustrates
this.

\includegraphics{Representativity-measure---working-paper-PDF_files/figure-latex/unnamed-chunk-2-1.pdf}

The balance/representativity within the continuous age covariate can
then be measured as \[
R = \frac{A}{A+B}
\]

The areas A and B are found by integrating the overall response rate
over the observed age values. \[
A+B = \int_{\text{age}} p_{\text{overall}} \, ds =  p_{\text{overall}} \left( \max \text{age} - \min \text{age} \right)
\] Let the age-specific response rate be given as a function of age
\$p\_\{\text{ age}\}( s) \$. The area A can be found as \begin{align}
A & = \int_{\text{age}} p_{\text{ age}}( s) 1_{ \{  p_{\text{ age}}( s) < p_{\text{overall}} \} } +  p_{\text{overall}} 1_{ \{  p_{\text{ age}} \geq p_{\text{overall}} \} } \, ds \\
&= \int_{\text{age}} p_{\text{overall}}+1_{ \{  p_{\text{ age}}( s) < p_{\text{overall}} \} }\left(p_{\text{ age}}( s) - p_{\text{overall}}  \right)  \, ds \\
&= p_{\text{overall}} \left( \max \text{age} - \min \text{age} \right) + 
\int_{\text{age}}1_{ \{  p_{\text{ age}}( s) < p_{\text{overall}} \} }\left(p_{\text{ age}}( s) - p_{\text{overall}}  \right)  \, ds,
\end{align} where
\(\int_{\text{age}}1_{ \{ p_{\text{ age}}( s) < p_{\text{overall}} \} }\left(p_{\text{ age}}( s) - p_{\text{overall}} \right) \, ds \leq 0\).

\newline

The representativity measure for a continuous covariate can then be
given as \begin{align}
R &= \frac{A}{A+B} \\
&= \frac{p_{\text{overall}} \left( \max \text{age} - \min \text{age} \right) + 
\int_{\text{age}}1_{ \{  p_{\text{ age}}( s) < p_{\text{overall}} \} }\left(p_{\text{ age}}( s) - p_{\text{overall}}  \right)  \, ds}{ p_{\text{overall}} \left( \max \text{age} - \min \text{age} \right)}\\
& = 1+\frac{
\int_{\text{age}}1_{ \{  p_{\text{ age}}( s) < p_{\text{overall}} \} }\left(p_{\text{ age}}( s) - p_{\text{overall}}  \right)  \, ds}{ p_{\text{overall}} \left( \max \text{age} - \min \text{age} \right)}.
\end{align}

\newline

Sometimes it is more convenient to take integer numbers instead of
continuous number and then transform the integral into a sum. Then the
expression changes to \begin{align}

R & = 1 + \frac{\sum_{i = \min \text{age}}^{\max \text{age}} 1_{ \{  p_{\text{ age}}( i) < p_{\text{overall}} \} } \left( p_{\text{ age}}( i) - p_{\text{overall}}  \right)}{p_{\text{overall}} \left( \max \text{age} - \min \text{age} \right)},
\end{align} where
\(p_{\text{ age}}( i) = \frac{ \sum_{j = \min \text{age}}^{\max \text{age}} 1_{ j = i}}{N}\)
and \(i\) is a given age value represented as an integer. The discrete
varion also applies to covariates that have clearly defined orders

\(\textbf{NB:}\) Confusion can arise with the presented measure since
the age specific response rate is not symmetric around the overall rate.
There is a different number of persons for each age which leads to
different weights to the overall response rate, and when considering the
representativity measure there is no control for a lack of persons
within a group, only if there is a lack of persons within a group in
regards of responding vs not responding.

\textbf{Empty} - Comparing several proportions using the chi square test

\newline

\end{document}
